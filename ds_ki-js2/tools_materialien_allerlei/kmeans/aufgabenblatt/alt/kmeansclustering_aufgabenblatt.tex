\documentclass[a4paper,11pt]{article}
\usepackage[utf8]{inputenc}
\usepackage[T1]{fontenc}
\usepackage[ngerman]{babel}
\usepackage{tikz}
\usepackage{amsmath}
\usepackage{geometry}
\geometry{margin=0.5in}

\newcommand{\kreis}{\tikz \draw[red, thick] (0,0) circle (4pt); }
\newcommand{\dreieck}{\tikz \draw[blue, thick] (0,0) -- (0.15,0.30) -- (0.30,0) -- cycle; }
\newcommand{\quadrat}{\tikz \draw[green!70!black, thick] (0,0) rectangle (0.3,0.3); }

\begin{document}

\title{\textbf{k-Means-Clustering mit Manhattan-Distanz}}
\author{}
\date{}

\maketitle

\section*{Einführung}

In dieser Übung wirst du den k-Means-Clustering-Algorithmus anwenden, um eine Menge von Datenpunkten in Cluster zu gruppieren. Dabei verwendest du die \textbf{Manhattan-Distanz} zur Berechnung der Distanzen zwischen Punkten und Zentroiden.

\section*{Aufgabe}

\begin{enumerate}
    \item \textbf{k-Means-Algorithmus durchführen}:
    \begin{enumerate}
        \item Berechne die Manhattan-Distanzen zwischen jedem Datenpunkt und den Zentroiden.
        \item Weise jeden Datenpunkt dem nächstgelegenen Zentroiden zu und markiere die Datenpunkte entsprechend ihrer Clusterzugehörigkeit, indem du sie mit einem Symbol umschließt:
        \begin{itemize}
            \item \kreis für Cluster 1
            \item \dreieck für Cluster 2
            \item \quadrat für Cluster 3
        \end{itemize}
        \item Aktualisiere die Position der Zentroiden, indem du den Mittelwert der X- und Y-Koordinaten der zugewiesenen Punkte berechnest.
        \item Wiederhole die Schritte a) bis c), bis sich die Clusterzuweisungen nicht mehr ändern.
    \end{enumerate}
\end{enumerate}

\section*{Datenpunkte und initiale Zentroiden}

\begin{minipage}{0.4\textwidth}
\begin{center}
\begin{tabular}{|c|c|c|}
\hline
\textbf{Punkt} & \textbf{X} & \textbf{Y} \\
\hline
$a$ & 2 & 10 \\ \hline
$b$ & 2 & 5  \\ \hline
$c$ & 8 & 4  \\ \hline
$d$ & 5 & 8  \\ \hline
$e$ & 7 & 5  \\ \hline
$f$ & 6 & 4  \\ \hline
$g$ & 1 & 2  \\ \hline
$h$ & 4 & 9  \\ \hline
$i$ & 6 & 2  \\ \hline
$j$ & 3 & 3  \\ \hline
$k$ & 5 & 6  \\ \hline
$l$ & 9 & 7  \\ \hline
\end{tabular}
\end{center}
\end{minipage}
&
\begin{minipage}{0.4\textwidth}
\begin{itemize}
    \item \textbf{\kreis Cluster 1, Zentroid $Z_{1}$}: $a (2, 10)$
    \item \textbf{\dreieck Cluster 2, Zentroid $Z_{2}$}: $d (5, 8)$
    \item \textbf{\quadrat Cluster 3, Zentroid $Z_{3}$}: $g (1, 2)$
\end{itemize}
\end{minipage}

\newpage
\begin{center}
\setlength{\tabcolsep}{1pt} % Weniger Platz innerhalb der Tabellenzellen
\begin{tabular}{l c l}

\begin{minipage}{0.4\textwidth}
  \vspace{-1em}
  \footnotesize 
  \begin{tabular}[b]{|c|c|c|c|c|}
    \hline
    \textbf{Punkt} & \textbf{D($Z_{1}$)} & \textbf{D($Z_{2}$)} & \textbf{D($Z_{3}$)} & \textbf{Cluster} \\
    \hline
    a & & & & \\ \hline
    b & & & & \\ \hline
    c & & & & \\ \hline
    d & & & & \\ \hline
    e & & & & \\ \hline
    f & & & & \\ \hline
    g & & & & \\ \hline
    h & & & & \\ \hline
    i & & & & \\ \hline
    j & & & & \\ \hline
    k & & & & \\ \hline
    l & & & & \\ \hline
    \end{tabular}
  \end{minipage}

\begin{minipage}{0.40\textwidth}
\begin{tikzpicture}[scale=0.5]
  \draw[step=1cm,gray,thin] (0,0) grid (10,11);
  \draw[thick,->] (0,0) -- (10.5,0) node[right] {\tiny X};
  \draw[thick,->] (0,0) -- (0,11.5) node[above] {\tiny Y};
  % X-Achse Beschriftung
  \foreach \x in {0,1,...,10} {
      \draw (\x,0) -- (\x,-0.2) node[below] {\tiny  \x};
  }
  % Y-Achse Beschriftung
  \foreach \y in {0,1,...,11} {
      \draw (0,\y) -- (-0.2,\y) node[left] {\tiny  \y};
  }
  \foreach \x/\y/\label in {
      2/10/$a$, 2/5/$b$, 8/4/$c$, 5/8/$d$, 7/5/$e$,
      6/4/$f$, 1/2/$g$, 4/9/$h$, 6/2/$i$, 3/3/$j$,
      5/6/$k$, 9/7/$l$}
  {
      \filldraw (\x,\y) circle (2pt) node[below right] {\footnotesize  \label};
  }
\end{tikzpicture}
\end{minipage}

\begin{minipage}{0.2\textwidth}
  \textbf{\footnotesize Zentroiden Iteration 2}
      \item \textbf{\kreis $Z_{1}$}: 
      \item \textbf{\dreieck $Z_{2}$}: 
      \item \textbf{\quadrat $Z_{3}$}: 
  \end{minipage}
  
\end{tabular}
\end{center}

% Rückkehr zu Standard-Seitenrändern
\restoregeometry

\end{document}
