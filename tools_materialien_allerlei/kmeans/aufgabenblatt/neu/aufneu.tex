\documentclass[a4paper,11pt]{article}
\usepackage[utf8]{inputenc}
\usepackage[T1]{fontenc}
\usepackage[ngerman]{babel}
\usepackage{amsmath}
\usepackage{geometry}
\usepackage{tikz}
\geometry{margin=1in}

% Cluster Symbole
\newcommand{\kreis}{\tikz \draw[red, thick] (0,0) circle (4pt); }
\newcommand{\dreieck}{\tikz \draw[blue, thick] (0,0) -- (0.15,0.30) -- (0.30,0) -- cycle; }
\newcommand{\quadrat}{\tikz \draw[green!70!black, thick] (0,0) rectangle (0.3,0.3); }

\begin{document}

\title{\textbf{k-Means-Clustering mit Manhattan-Distanz}}
\author{}
\date{}

\maketitle

\section*{Einführung}
In dieser Übung wenden Sie den k-Means-Clustering-Algorithmus an, um eine Menge von Datenpunkten in Cluster zu gruppieren. Der Algorithmus verwendet die \textbf{Manhattan-Distanz} zur Berechnung der Distanzen zwischen Punkten und Zentroiden.

\section*{Datenpunkte und initiale Zentroiden}

Die Tabelle zeigt die Ausgangsdatenpunkte sowie die initialen Zentroiden. Jeder Punkt ist mit einem Buchstaben von $a$ bis $l$ gekennzeichnet.

\begin{minipage}{0.45\textwidth}
\begin{center}
\begin{tabular}{|c|c|c|}
\hline
\textbf{Punkt} & \textbf{X} & \textbf{Y} \\
\hline
$a$ & 2 & 10 \\ \hline
$b$ & 2 & 5  \\ \hline
$c$ & 8 & 4  \\ \hline
$d$ & 5 & 8  \\ \hline
$e$ & 7 & 5  \\ \hline
$f$ & 6 & 4  \\ \hline
$g$ & 1 & 2  \\ \hline
$h$ & 4 & 9  \\ \hline
$i$ & 6 & 2  \\ \hline
$j$ & 3 & 3  \\ \hline
$k$ & 5 & 6  \\ \hline
$l$ & 9 & 7  \\ \hline
\end{tabular}
\end{center}
\end{minipage}
\hfill
\begin{minipage}{0.45\textwidth}
\textbf{Initiale Zentroiden:}
\begin{itemize}
    \item \textbf{\kreis Cluster 1, Zentroid $Z_{1}$}: $(2, 10)$
    \item \textbf{\dreieck Cluster 2, Zentroid $Z_{2}$}: $(5, 8)$
    \item \textbf{\quadrat Cluster 3, Zentroid $Z_{3}$}: $(1, 2)$
\end{itemize}
\end{minipage}

\section*{Aufgabe}
\begin{enumerate}
    \item Berechnen Sie die Manhattan-Distanzen jedes Punktes zu den Zentroiden $Z_{1}$, $Z_{2}$ und $Z_{3}$.
    \item Ordnen Sie jeden Punkt dem Cluster mit der kleinsten Distanz zu. Ist die Distanz gleich, wird der erste Zentroid bevorzugt.
    \item Aktualisieren Sie die Zentroidenpositionen, indem Sie den Mittelwert der X- und Y-Koordinaten der Punkte in jedem Cluster berechnen (auf ganze Zahlen gerundet).
    \item Wiederholen Sie die Schritte, bis sich die Clusterzuweisungen nicht mehr ändern.
\end{enumerate}

\section*{Iteration 1 (vorgefüllt)}

\begin{tabular}{|c|c|c|c|c|}
\hline
\textbf{Punkt} & \textbf{D($Z_{1}$)} & \textbf{D($Z_{2}$)} & \textbf{D($Z_{3}$)} & \textbf{Cluster} \\
\hline
$a$ & 0 & 5 & 9 & \kreis $Z_{1}$ \\ \hline
$b$ & 5 & 6 & 4 & \quadrat $Z_{3}$ \\ \hline
$c$ & 10 & 7 & 10 & \dreieck $Z_{2}$ \\ \hline
$d$ & 5 & 0 & 10 & \dreieck $Z_{2}$ \\ \hline
$e$ & 10 & 5 & 9 & \dreieck $Z_{2}$ \\ \hline
$f$ & 11 & 6 & 7 & \dreieck $Z_{2}$ \\ \hline
$g$ & 9 & 10 & 0 & \quadrat $Z_{3}$ \\ \hline
$h$ & 3 & 2 & 10 & \kreis $Z_{1}$ \\ \hline
$i$ & 12 & 7 & 5 & \quadrat $Z_{3}$ \\ \hline
$j$ & 7 & 7 & 3 & \quadrat $Z_{3}$ \\ \hline
$k$ & 9 & 3 & 7 & \dreieck $Z_{2}$ \\ \hline
$l$ & 14 & 7 & 13 & \dreieck $Z_{2}$ \\ \hline
\end{tabular}

\textbf{Neue Zentroiden:}
\begin{itemize}
    \item \textbf{\kreis Cluster 1, Zentroid $Z_{1}$}: $(3, 10)$
    \item \textbf{\dreieck Cluster 2, Zentroid $Z_{2}$}: $(7, 6)$
    \item \textbf{\quadrat Cluster 3, Zentroid $Z_{3}$}: $(3, 3)$
\end{itemize}

\section*{Iteration 2 (auszufüllen)}

\begin{tabular}{|c|c|c|c|c|}
\hline
\textbf{Punkt} & \textbf{D($Z_{1}$)} & \textbf{D($Z_{2}$)} & \textbf{D($Z_{3}$)} & \textbf{Cluster} \\
\hline
$a$ & & & & \\ \hline
$b$ & & & & \\ \hline
$c$ & & & & \\ \hline
$d$ & & & & \\ \hline
$e$ & & & & \\ \hline
$f$ & & & & \\ \hline
$g$ & & & & \\ \hline
$h$ & & & & \\ \hline
$i$ & & & & \\ \hline
$j$ & & & & \\ \hline
$k$ & & & & \\ \hline
$l$ & & & & \\ \hline
\end{tabular}

\textbf{Neue Zentroiden:}
\begin{itemize}
    \item \textbf{\kreis Cluster 1, Zentroid $Z_{1}$}: \_\_\_\_
    \item \textbf{\dreieck Cluster 2, Zentroid $Z_{2}$}: \_\_\_\_
    \item \textbf{\quadrat Cluster 3, Zentroid $Z_{3}$}: \_\_\_\_
\end{itemize}

\section*{Iteration 3 (auszufüllen)}

Führen Sie die Berechnungen weiter durch, bis sich die Zentroiden nicht mehr ändern.

\end{document}
